\documentclass[article]{siamltex}
\usepackage{epsfig}
\usepackage{subfigure,color,graphicx,cite}
\usepackage{amsfonts}
\usepackage{amsmath}
\usepackage{epstopdf}
\input psfig.sty

\textheight 220 true mm \textwidth 150 true mm \topmargin -10mm
\oddsidemargin 0cm \evensidemargin 0cm \leftskip  0mm \rightskip
0mm \hoffset 5mm

%\newenvironment{proof}{\begin{trivlist}
%    \item[\hskip\labelsep{\it Proof.}]}{$\hfill\Box$\end{trivlist}}

\newtheorem{Theorem}{Theorem}
\newtheorem{Corollary}{Corollary}
\newtheorem{Lemma}{Lemma}
\newtheorem{Remark}{Remark}
\newtheorem{Definition}{Definition}
\newtheorem{Example}{Example}
%\newtheorem{alg}[Example]{Algorithm}{\bf}%{\rm}
%\newtheorem{program}[Example]{Program}{\bf}{\rm}

\allowdisplaybreaks

%\renewcommand{\qedsymbol}{{ \vrule height7pt width7pt depth0pt}\par\bigskip}

\def\lq{\left(}
\def\rq{\right)}

\def\cala{{\mathcal A}}
\def\bx{\boldsymbol x}
\def\by{\boldsymbol y}
\def\bz{{\boldsymbol z}}
\def\be{{\boldsymbol e}}
%\def\bk{\boldsymbol k}
\def\bk{{\bf k}}
\def\bK{\boldsymbol K}
\def\bm{{\bf m}}
%\def\bm{\boldsymbol m}
\def\bw{{\boldsymbol w}}
\def\bW{{\boldsymbol W}}
\def\bh{{\boldsymbol h}}
\def\bg{{\boldsymbol g}}
\def\bv{{\boldsymbol v}}
\def\bu{{\boldsymbol u}}
\def\bDel{{\boldsymbol\Delta}}
\def\ba{{\boldsymbol a}}
\def\bA{{\boldsymbol A}}
\def\bb{{\boldsymbol b}}
%\def\bi{{\boldsymbol i}}
%\def\bj{{\boldsymbol j}}
%\def\bl{{\boldsymbol l}}
\def\bi{{\bf i}}
\def\bj{{\bf j}}
\def\bl{{\bf l}}
\def\btau{{\boldsymbol \tau}}
\def\bthe{{\boldsymbol \theta }}
\def\bmu{{\boldsymbol \mu }}
\def\bxi{{\boldsymbol \xi }}
\def\bnu{{\boldsymbol \nu }}
\def\blam{{\boldsymbol \lambda }}
\def\bgam{{\boldsymbol \gamma }}
\def\bsigma{{\boldsymbol \sigma }}
\def\bzeta{{\boldsymbol \zeta }}
\def\bomega{{\boldsymbol \omega }}
\def\balpha{{\boldsymbol \alpha}}
\def\bbeta{{\boldsymbol \beta}}
\def\bzero{{\boldsymbol 0}}

\def\bone{{\boldsymbol 1}}
\def\calV {\mathcal {V}}

\def\calH {\mathcal {H}}
\def\K {\mathcal {K}}
\def\calP {\mathcal {P}}
\def\calF {\mathcal {F}}
\newcommand{\trig}{{\rm trig}}
\newcommand{\df}[2]{\displaystyle\frac{#1}{#2}}
\newcommand{\comb}[2]{{{#1}\choose {#2}}}
\newcommand{\crno}{\cr\noalign{\vskip2mm}}
\newcommand{\Enorm}[1]{\left\Vert#1\right\Vert_{E,\omega}}
\newcommand{\norm}[1]{\left\Vert#1\right\Vert}

\newcommand{\Z}{{\mathbb Z}}
\newcommand{\R}{\mathbb{R}}
\newcommand{\N}{\mathbb{N}}
\newcommand{\per}{\text{per}}
\newcommand{\fil}{{fil}}
\newcommand{\dprod}{\displaystyle\prod}
\newcommand{\dsum}{\displaystyle\sum}
\newcommand{\dint}{\displaystyle\int}
\newcommand{\sign}{\text{sign}}
\newcommand{\abs}[1]{\left\vert#1\right\vert}
%\newcommand{\err}{\text{err}}

\DeclareMathOperator*{\err}{err} \DeclareMathOperator*{\avg}{avg}
\DeclareMathOperator*{\argmin}{argmin}
\DeclareMathOperator*{\trace}{trace}
%\DeclareMathOperator*{\diag}{diag}
\DeclareMathOperator*{\argmax}{argmax}
\DeclareMathOperator*{\minimize}{minimize}
\DeclareMathOperator*{\subjectto}{subject \,\,to:}
%%%%%%%%%%%%%%%%%%%%%%%%%%%%%%%%%%%%%%%%%%%%%%%%%%%%%%%%%%%%%%%%%%%%%%%%%%%%%
%%%%%%%%%%%%%%%%%%%%%%%%%%%%%%%%%%%%%%%%%%%%%%%
\begin{document}




%%--------------------------------------------------




\title{Chebyshev Spectral Methods on Lattices for High-Dimensional
Approximation \thanks{This research was supported by NSFC 11701358 .}}


\author{
Xiaoyan Zeng\thanks{Department of  Mathematics,
 Shanghai University, Shanghai 200444, P.R.China ({\tt
cherryzxy@shu.edu.cn}).}\and
Fred J. Hickernell\thanks{Department of Applied Mathematics,
 Illinois Institute of Technology, Chicago, IL 60616, U.S.A. ({\tt
hickernell@iit.edu}).}
}

\date{}
\maketitle

\begin{abstract}
The aim of this paper is to develop numerical methods that can
avoid the curse of the dimensionality of the domain for
approximation problems. It is assumed that $f$ can be expressed by
an absolutely convergent multidimensional Chebyshev series
expansion. The design points set, $P$, considered here, in
contrast to the classic ones, contains only half of the node sets
of integration lattices with a cosine transformation. Its
Chebyshev coefficients are approximated at some selected
nonnegative wavenumbers by  quadrature rules and all other
Chebyshev coefficients are set to be zero. The approximation of
$f$ is then taken to be the Chebyshev expansion based on these
approximated Chebyshev coefficients. The error of the
approximation is discussed in the weighted $L_2$ norm.  We prove
the convergence rate is independent of the dimension of the domain
and the convergence rate is related to the rate of the true
Chebyshev coefficients' decay.
\end{abstract}

\begin{keywords}
 Chebyshev spectral methods, integration lattices, tractability
\end{keywords}

\begin{AMS}
Primary 65N35, 65T40;~Secondary 65D05, 65D32
\end{AMS}
\section{Introduction} In this paper, we will present a Chebyshev
spectral method on lattices for the approximation of non-periodic
functions with domain $\mathcal X=[-1,1]^d.$ Spectral methods
became popular in the mid-1970s. There are many papers and
monographs that have significantly contributed to the state of the
art, e.g., \cite{BernardiM1997}, \cite{Boyd2000},
\cite{CanutoHQZ1988}, \cite{Don1994}, \cite{Fornberg1995}, \cite{Gottlieb1984},
\cite{GottliebO1977}, \cite{Mercier1989},
 \cite{ShenT07},\cite{Trefethen2000}
 and the references listed in them.


Suppose that $f$, defined in the domain $\mathcal X$, can be
expressed by an absolutely convergent multidimensional Chebyshev
series expansion:
\begin{equation}\label{cheb}
  f(\bx)=\dsum_{\bk \in {\N}^d_{0}}F(\bk)T_{\bk}(\bx),
\end{equation}
 where ${\N}^d_{0}:=\{\bk=(k_1,\cdots,k_d)^T \in {\Z}^d:\ k_j\geq {0}\}$ and
\begin{equation}\label{ChebPoly}
  T_{\bk}(\bx)=\dprod_{j=1}^{d}T_{k_j}(x_j)=\dprod_{j=1}^{d}
   \cos(k_j\arccos x_j).
\end{equation}
The Chebyshev coefficients $F(\bk)$ can be derived by using the
orthogonality relations of $T_{\bk}(\bx)$ with respect to the
weight function $\omega(\bx)=\dprod_{j=1}^{d}
\omega_j(x_j)=\dprod_{j=1}^{d}{\frac{1}{\sqrt{1-x_j^2}}}:$
$$
F(\bk)=\frac{c(\bk)}{\pi^d} \dint_{(-1,1)^d} f(\bx)T_{\bk}(\bx)
\omega(\bx)\;d\bx,
$$
where
$$
 c(\bk):=2^{\sharp \{k_j\neq 0\}}.
$$
Because $T_{\bk}(\bx)$ can be written as in \eqref{ChebPoly} and
cosine function is an even function, the Chebyshev polynomial and
Chebyshev coefficients can be extended to the whole integral
vector
space $\Z^d$ %so that
and thus
  \begin{equation}\label{k-k}
  F(\bk)=F(\abs{\bk})\quad \hbox{and}\quad
 T_{\bk}(\bx)=T_{\abs{\bk}}(\bx)
 \qquad \hbox{for all}\ \bk\in {\Z}^d,
  \end{equation}
  where $|\bk|=(|k_1|,\ldots,|k_d|)^T$.


After the change of variables by $\bx=\cos(\pi\bthe):=
\left(\cos(\pi\theta_1),\ldots, \cos(\pi\theta_d)\right)^T$, we
obtain
\begin{equation}F(\bk)={c(\bk)} \dint_{(0,1)^d} f(\cos(\pi\bthe))\dprod_{j=1}^d\cos(\pi k_j\theta_j)\;d\bx. \end{equation}
% the
%integrands defining $F(\bk)$ are periodic in
%$\bthe:=\left(\theta_1,\ldots,\theta_d\right)$ and therefore can
%be well approximated by the lattice rules (see e.g.,
%\cite{Hickernell1998}, \cite{SloanJoe1994}).
One then chooses a wavenumber space $\K \in {\N}^d_0$ and finds a
quadrature rule  to approximate  $F(\bk)$ for
$\bk\in\K$. % see also (\ref{hatF}) for the definition of $\hat
%F(\bk)$.
Denote the approximation to $F(\bk)$ by $\hat F(\bk).$ The
function $f$ can therefore be approximated by
\begin{equation}\label{fhatK}
\hat{f}(\bx):=\dsum_{\bk\in {\K}}\hat{F}(\bk)T_{\bk}(\bx).
\end{equation}
Obviously, the approximation error depends on how close $\hat
F(\bk)$ is to $F(\bk)$ for $\bk$ in the wavenumber space $\K$ and
how fast $F(\bk)$ decays with respect to $\bk$ as $\bk$ gets away
from $\K$. To minimize the approximation error, one needs to
understand how it depends on design points and $\K$.

The contents of the chapter are outlined as the following.
Integration lattice is introduced in Section 2.  %The Chebyshev
%spectral method on lattices is described in Section 3.
 A formula for the error of this method
and the worst-case error analysis are  presented in  Section 3. In
Section 4, the tractability and strong tractability are discussed
in weighted space. The numerical simulation result of the
8-dimensional borehole function is provided in Section 5.

\vskip5mm
\section{Integration Lattice}
An d-dimensional integration lattice, $L$, is a discrete subset of
$\R^d$ that is closed under addition and subtraction and which
contains the set of integer vectors $\Z^d$ as a subset. The node
set for a lattice, $L$, is the set of points in the lattice that
fall inside the unit cube, i.e., $\calP:=L\cap [0,1)^d$. A shifted
lattice is defined as $\L_{\mathbf{\Delta}}:= \{
\bz+\mathbf{\Delta}:\ \bz\in L\}$, where the shifted
$\mathbf{\Delta}\in [0,1)^d$ is often chosen randomly to eliminate
the bias in the algorithms. Grids aligned with the coordinate axes
are rank-d lattices, since their node sets require $d$ generators.
On the other hand, rank-1 lattices, whose node sets need only a
single generator, are more commonly used to solve high dimensional
problems. The readers may wish to consult \cite{Hickernell1998},
\cite{HickernellNie2003}, \cite{HuaWang1981},
\cite{Niederreiter1992}, and \cite{SloJoe92} for more background
materials of integration lattices.

\smallskip

A few examples of integration lattices are given below and the
figures of two-dimensional examples are also provided in Fig.
\ref{nodeset} and Fig.\ref{duallattice}.
\begin{Example}\label{exam1}
Two dimensional lattices with $144$ points:
\begin{enumerate}
\item A rectangular grid, its node set (Fig.~\ref{nodeset-a}) and
its dual lattice  (Fig.~\ref{duallatice-a}):
\begin{subequations}
\begin{eqnarray}
&&L:=\{(i/12, j/12):\ i, j\in {\Z}\},\label{ga}\\
&&\calP:=\{(i/12, j/12):\ i, j=0,1,2,\ldots, 11\},\label{gb}\\
&&L^{\bot}:=\{12( i, j )  :\ i, j\in {\Z}\}.\label{gc}
\end{eqnarray}
\end{subequations}
\item A rank-$1$ lattice, its node set (Fig.\ref{nodeset-c}) and
its dual lattice (Fig.\ref{duallatice-c}):
\begin{subequations}
\begin{eqnarray}
&&L:=\{(1,89)i/144 +(0,1)j:\ i, j\in {\Z}\},\label{La}\\
&&\calP:=\{(1,89)i/144\quad\hbox{mod}\ 1:\ i=0,\ldots, 143\},\label{Lb}\\
&&L^{\bot}:=\{(144,0)i+(-89,1)j:\ i, j\in {\Z}\}.\label{Lc}
\end{eqnarray}
\end{subequations}
\item Shifted lattice ($L$ is defined as in \eqref{La}) with
$\mathbf{\Delta}=(1,89)/288$ and its node
set (Fig.\ref{nodeset-d}):
\begin{subequations}
\begin{eqnarray}
&&L_{\mathbf{\Delta}}:=L + \mathbf{\Delta},\label{sa}\label{s3a}\\
&&\calP:=\{(1,89)i/144+\mathbf{\Delta} \quad\hbox{mod}\ 1:\
i=0,\ldots, 143\}.\label{sb}%,\\
%&&L^{\bot}:=\{(144,0)i+(-89,1)j:\ i, j\in {\Z}\}.
\end{eqnarray}
\end{subequations}
\end{enumerate}
\end{Example}

\begin{Example}
Three dimensional lattices with $2^9$ points:
\begin{enumerate}
\item A rectangular grid, its node set and its dual lattice:
\begin{subequations}
\begin{eqnarray}
&&L:=\{(i/2^{3}, j/2^{3}, k/2^{3}):\ i, j, k
\in {\Z}\},\label{g3a}\\
&&\calP:=\{(i/2^{3}, j/2^{3}, k/2^{3}):\ i, j,
k=0,1,2,\ldots, 2^{3}-1\},\label{g3b}\\
&&L^{\bot}:=\{ (i, j ) 2^3 :\ i, j,k\in {\Z}\}.\label{g3c}
\end{eqnarray}
\end{subequations}

\item A rank-$1$ lattice, its node set and its dual lattice:
\begin{subequations}
\begin{eqnarray}
\centering &&L:=\{(1,249,249^2)i/2^{9}+ (0,1,0)j+ (0,0,1)k:\
i, j, k\in {\Z}\},\label{L3a}\\
&&\calP:=\{(1,249,249^2)i/2^{9}\quad\hbox{mod}\ 1:\ i=0,\ldots,
2^{9}-1\},\label{L3b}\\
&&L^{\bot}:=\{(2^9,0,0)i +(-249,1,0 )j+( -249^2,0,1 )k: i,j,k\in
{\Z}\}.\label{L3c}
\end{eqnarray}
\end{subequations}
\item Shifted lattice ($L$ is defined as in \eqref{L3a}) with
$\mathbf{\Delta}
=(1,249,249^2)/2^{10}$ and its node set :%and its dual lattice:
\begin{subequations}
\begin{eqnarray}
&&L_{\mathbf{\Delta}}:=L + \mathbf{\Delta},\\
&&\calP:=\{(1,249,249^2)i/2^{9}+ \mathbf{\Delta} \quad\hbox{mod}\
1:\ i=0,\ldots, 2^{9}-1\}.
\end{eqnarray}
\end{subequations}
\end{enumerate}
\end{Example}


\begin{figure}
\centering \subfigure[Node Sets in
\eqref{gb}]{\label{nodeset-a}\includegraphics[width=7cm,height=7cm]{./eps/grid.eps}}\\
\subfigure[Node Set in
\eqref{Lb}]{\label{nodeset-c}\includegraphics[width=7cm,height=7cm]{./eps/goodlattice.eps}}\\
\subfigure[Node Set in
\eqref{sb}]{\label{nodeset-d}\includegraphics[width=7cm,height=7cm]{./eps/shiftgoodlattice.eps}}
 \caption{Node set of  some two-dimensional integration lattices in Example \ref{exam1}.} \label{nodeset}
\end{figure}


\begin{figure}
\centering
\subfigure[Dual Lattice in \eqref{gc}]{\label{duallatice-a}\includegraphics[width=7cm,height=7cm]{eps/dualforgrid.eps}}\\
\subfigure[Dual Lattice in
\eqref{Lc}]{\label{duallatice-c}\includegraphics[width=7cm,height=7cm]{eps/dualforgoodlattice.eps}}
 \caption{Dual lattice of  some two-dimensional lattices in Example \ref{exam1}.} \label{duallattice}
\end{figure}

If one projects the points of grids in the half unit cube, as
shown in Fig.\ref{nodeset-a}, onto $x_1$ axis ($x_2$ axis), there
are only $\sqrt{144}=12$ points falling on the axis while for the
node set of rank-$1$ lattice, there are $144$ projected points on
each axis. Generally speaking, in every $s$-dimensional subspace,
$1\leq s\leq d,$ for a $d$-dimensional rectangular grid, there are
only $n^{s/d}$ projected points on each axis, while for the node
set of rank-$1$ integration lattice, there are always $n$
projected points. This is one of the features which make rank-$1$
integration lattices outperform rectangular grids.

The set of all integer vectors, ${\Z}^d$, may be written as
${\calV}_n\oplus L^\bot$, where $\oplus$ denotes the direct sum,
and ${\calV}_n$ is a set of $n$ $d$-dimensional integer vectors
having the property that any two distinct vectors in ${\calV}_n$
differ by a vector that is not in the dual lattice, i.e.,
\begin{equation}                                 \label{calV-condition-1}
\bnu,\bmu\in{\calV}_n \text{ and } \bnu\neq\bmu
 \quad\Rightarrow \quad \bnu-\bmu\notin L^\bot.
\end{equation}
Assume that $\bzero\in{\calV}_n.$ There are many ways to choose
${\calV}_n.$  We can define a ``size" function of  wavenumbers and then use it to identify the appropriate wavenumbers among sets of wavenumbers. One most used   ``size"  function $r,$ is deifned as below
\begin{equation}\label{r}r_\bgam(\bnu)=\dprod_{j=1}^d
r_{\gamma_j}(\nu_j),\end{equation}
 where $$r_{\gamma_j}(\nu_j) =
 \left\{\begin{array}{l}
1 \ \ \ \ \ \ \ \,\mbox{if } \,\nu_j=0, \\
\frac{|\nu_j|}{\gamma_j} \ \ \ \mbox{otherwise };
\end{array}\right. $$
where ${\boldsymbol \gamma}=(\gamma_1, \gamma_2,\cdots)$with  $
1\ge \gamma_1\ge \gamma_2\ge \cdots\ge 0.$  Note that  $r$
function is defined slight differently from the literature.
\cite{KuoSloWoz05a} used $r_{\alpha}(\bgam,\bnu),$ where
$r_{\alpha}(\bgam,\bnu)=[r_{\bgam^{-1/\alpha}}(\bnu)]^\alpha.$
Please be careful when refer to the paper.

One shortcoming of  the above ``size" function is that $0$ and $1$ take the same ``size", which is clearly not the case. To differentiate  $0$ and $1$, we define a new ``size" function as below,
\begin{equation}\label{rnew}\tilde{r}_\bgam(\bnu)=\dprod_{j=1}^d
\tilde{r}_{\gamma_j}(\nu_j),\end{equation}
 where \begin{equation}\tilde{r}_{\gamma_j}(\nu_j) =\frac{|\nu_j|}{\gamma_j}+1. \end{equation}
 Clearly, $1 \le \tilde{r}_{\gamma_j}(\nu_j)/ r_{\gamma_j}(\nu_j)\le 1+\gamma_j$, and $\tilde{r}_{\gamma_j}(\nu_j)/ r_{\gamma_j}(\nu_j)=1,$
if $\nu_j \rightarrow \infty,$ for all $j=1,\cdots,d.$


With the newly defined ``size" function $\tilde{r}_\bgam(\bnu)$,  ${\calV}_n$ is then chosen to satisfy
 \begin{equation}                      \label{calV-condition-2}
\tilde{r}_{\bgam}(\bnu)\leq \tilde{r}_{\bgam}(\bnu+\blam), \quad
 \forall\bnu\in{\calV}_n, \quad \forall\blam\in{L^\bot}.
 \end{equation}
 This may be done by ranking wave numbers $\bnu\in{\Z^d}$ in terms of
 their $\tilde{r}_{\bgam}(\bnu)$ values, and then choosing the first $n$ distinct wave numbers satisfying the condition
 \eqref{calV-condition-1}.
% have distinct values of $\bnu$.crospongding
In particular, if node set of  rank-$1$
 lattice is used,  $\bnu_i \in \calV_n,\, i=0,\cdots,n-1$ are chosen to satisfy
 condition (\ref{calV-condition-2}) and
  \begin{equation}                      \label{calV-condition-3}
 \bnu_i \cdot \bh =i \mod n.
 \end{equation}
With the definition of $\calV_n,$ define
\begin{eqnarray}                            \label{tilde D}
 \widetilde{D}_{\alpha,q,\bgam}({\calV}_n)&=&\Big\|
  \df{1_{\bmu\notin{\calV}_n}}{[\tilde{r}_{\bgam}(\bmu)]^{\alpha}}\Big\|_q\nonumber\\&=&\left\{
  \begin{array}{ll}
    \left(\dprod_{j=1}^d\gamma_j^{\alpha q}\left(2\bzeta(\alpha q,\gamma_j )-1\right)-\left\|\df{1_{\bnu\in{\calV}_n}}{[\tilde{r}_{\bgam}(\bnu)]^{\alpha}}
    \right\|_q^q\right)^{1/q},
   & 1<q<\infty,
   \crno
   \sup\limits_{\bmu\notin{\calV}_n}[\tilde{r}_{\bgam}(\bmu)]^{-\alpha}, & q=\infty,
   \end{array} \right.
\end{eqnarray}
The two-variable function $\bzeta$ in the above equation is the Hurwitz zeta function. When the second variable equals to $1$, it becomes Riemann zeta function and hence normally is written as a one-variable function. (see \cite{Hickernell1998} and \cite{SloJoe92}).

It is obviously true that $\tilde{r}_{\gamma}(\nu)\ge {r}_{\gamma}(\nu)$.  Hence by similar proof as in  \cite[Lemma 1.13]{Zengthesis},  the following lemma holds.

\vskip5mm   \begin{Lemma}\label{CBC} Suppose $\alpha q>1$ and $n$
is large enough. Then a generating vector of a rank-$1$
integration lattice can be obtained  by component-by-component
construction  such that
\begin{equation}\label{eqnBCB}\widetilde{D}_{\alpha, q,\bgam}({\calV}_n)= \mathcal{O}\left(
{n^{-\frac{\alpha-1/q}{2-1/(\alpha q)} +\epsilon}}\right).
\end{equation} The big $\mathcal{O}$-notation in \eqref{eqnBCB} dependent on $d.$
 \noindent Let $1/q\le v <\alpha.$  If $\dsum_{j=1}^{+\infty} \gamma_{j}^{\alpha /v}<+\infty,$ then
\begin{equation}\label{imSTRsBCB}\widetilde{D}_{\alpha,
q,\bgam}({\calV}_n)
=\mathcal{O}\left(n^{-\frac{v-1/q}{2-1/(qv)}}\right),\end{equation}
if
$a:=\lim\sup_{d\rightarrow\infty}\sum\limits_{j=1}^{d}\gamma_j^{{
\alpha/v}}/\log(d+1)<\infty,$ then
\begin{equation}\label{imTRsBCB}
\widetilde{D}_{\alpha, q,\bgam}({\calV}_n)= \mathcal{O}\left(
d^{b}{n^{-\frac{v-1/q}{2-1/(qv)}}}\right).\end{equation} The
implied factor in the big $\mathcal{O}$-notations in
\eqref{imSTRsBCB} and \eqref{imTRsBCB} are independent of $d$. The
exponent $b$ is any positive number greater than $2va\bzeta({
\alpha/v})$ and $\frac{v-1/q}{2-1/(qv)}$ can be arbitrarily close
to $\frac{\alpha-1/q}{2-1/(\alpha q)}.$
\end{Lemma}




\vskip8mm  \section{Chebyshev Spectral Methods on Lattices}

Define  variation of a function $f$ as
\begin{equation}\label{V-beta-f}
 V_{\alpha,p,\bgam}(f):=  \norm{F(\bk)[\tilde{r}_\bgam(\bk)]^{\alpha}1_{\bk\in {\N}^d_0}}_p, p\ge 1.
 \end{equation}
Variation of a function is a measure of the variability of a
function. The parameter $\alpha$ is used to characterize how fast
the Chebyshev coefficients decay.
 Furthermore, the function space ${\calF}_{\alpha,p,\bgam}:=\{f:\,\,
  V_{\alpha,p,\bgam}(f)< \infty\}$ consists of all functions with finite variation. The weight $\gamma_j$ of the Banach space $F_{\alpha,p,\bgam}$
moderates the behavior of functions with respect to the $j$th
variable. Small $\gamma_j$ means that functions depend weakly on
the $j$th variable. Functions concerned in this paper are
demanded in the function space ${\calF}_{\alpha,p,\bgam}.$
The variation used in literature is normally defined by $\norm{F(\bk)[{r}_\bgam(\bk)]^{\alpha}1_{\bk\in {\N}^d_0}}_p$. The two variations are equivalent, because the ratio of these two variations (with $V_{\alpha,p,\bgam}(f)$ defined in \eqref{V-beta-f} as the numerator) is no less than $1$ and no greater than $\prod_{j=1}^d(1+\gamma_j)^\alpha$.




To approximate the Chebyshev coefficients, the node set  used here
is a little different from the classic one introduced in Section
1, since the points will fall onto $[-1, 1]^d$ after a cosine
transformation. Let the $n$-cardinality node set of a shifted
lattice be
$$
P:=\{i\bh/n+ \mathbf{\Delta},\ i=0,\ldots, n-1\},
$$
where the $d$-vector $\bh$ is a rank-$1$ lattice generator and
$\mathbf{\Delta} =\bh/(2n)$. In other words, the node set is a
 set consisting of points generated by $\bh/(2n)$ times odd integers from $1$ to
$2n-1.$ Define the dual lattice
$$
L^{\bot}:=\{\bl\in {\Z}^d:\quad \bl\cdot \bh=0 \quad\hbox{mod}\
2n\}.
$$
  In
Fig.~3.1, the left panel shows what  $P$ looks like after cosine
transformation with $n= 64 $ and $h=(1 \,\,\, 75)^T;$ the right
panel displays $64$ grid points in the half open unit cube  after
a cosine transformation.
\begin{figure}[htbp]\label{NewCheblattice}
\begin{center}
\includegraphics[height=7cm, width=7cm, angle=0] {eps/NewChebgrid.eps}
\includegraphics[height=7cm, width=7cm, angle=0]{eps/NewCheblattice.eps}
\caption{Left:$64$ rectangular grid points after cosine transformation ; right: set $P$ with $n= 64 $ and $h=(1 \,\,\, 75)^T$ after cosine
 transformation. }
\end{center}
\end{figure}
The reason to choose $P$ and the dual lattice in this way is
because of the following lemma. \vskip5mm   \begin{Lemma} With the
above definitions of node set $P$ and dual lattice $L^{\bot}$, it
holds that
\begin{equation}\label{shiftlatticeproperty}
\frac{1}{n} \dsum_{\bz\in  P} \cos(\pi \bl\cdot \bz)= \left\{
\begin{array}{ll}
(-1)^{\bl\cdot \bh/(2n)},&
\bl\in L^{\bot};\\
0,& \bl\notin L^{\bot}.
\end{array}
\right.
\end{equation}
\end{Lemma}
\begin{proof} Consider the sum, ${\displaystyle \frac{1}{n}
\dsum_{\bz\in P} e^{\imath \pi \bl\cdot \bz}}$, where
$\imath=\sqrt{-1}$. If $\bl\notin L^{\bot}$, i.e., $\bl\cdot
\bh\neq 0\ \hbox{mod}\ 2n$, then
\begin{eqnarray}\label{thesum}
\frac{1}{n} \dsum_{\bz\in P} e^{\imath \pi \bl\cdot \bz}
 &=& \frac{1}{n} \dsum_{i=0}^{n-1} e^{\imath \pi (2i+1)\bl
\cdot \bh/(2n)}= \frac{1}{n} \frac{e^{\imath \pi \bl \cdot
\bh/(2n)} - e^{\imath \pi (2n+1)\bl \cdot \bh/(2n)}}{1- e^{\imath
\pi \bl \cdot
\bh/n}}\nonumber\\
&=& \frac{1}{n} \frac{1- (-1)^{\bl\cdot \bh}} {e^{-\imath \pi \bl
\cdot \bh/(2n)}- e^{\imath \pi \bl \cdot \bh/(2n)}}
 =
\frac{\imath(1-(-1)^{\bl\cdot\bh})}
{2n\sin(\pi\bl\cdot\bh/(2n))}.
\end{eqnarray}
%And it
This means that the real part of the sum is zero if $\bl\notin
L^{\bot}$, that is
\begin{equation}
\frac{1}{n} \dsum_{\bz\in P} \cos({\pi \bl\cdot \bz})
 =0.
\end{equation}
If $\bl\in L^{\bot}$, i.e., $\bl\cdot \bh = 0\ \hbox{mod}\ 2n$,
from (\ref{thesum}) then
\begin{equation}
\frac{1}{n} \dsum_{\bz\in P} e^{\imath \pi \bl\cdot \bz}
 =(-1)^{\bl\cdot \bh/(2n)}.
\end{equation}
Hence (\ref{shiftlatticeproperty}) follows.   \end{proof}

\vskip5mm

 Change the variable $\bx=(\cos(\pi \theta_1), \ldots, \cos(\pi
\theta_d))^T$. The expansion (\ref{cheb}) is then equivalent to
\begin{eqnarray}
 f(\bx)&=& \dsum_{\bk\in {\N}^d_{0}} F(\bk) \dprod_{j=1}^{d} \cos(\pi
 k_j\theta_j)\nonumber\\
  &=& \dsum_{\bk\in {\N}^d_{0}} F(\bk) \dprod_{j=1}^{d}
  \frac{e^{\imath \pi k_j\theta_j}
  + e^{-\imath \pi k_j \theta_j}}{2}\nonumber\\
  &=& \dsum_{\bk\in {\Z}^d} \frac{F(|\bk|)}{c(\bk)} e^{\imath \pi
  \bk\cdot \bthe},\qquad\hbox{where}\
  |\bk|=(|k_1|,\ldots,|k_d|)\nonumber\\
  &=& \dsum_{\bk\in {\Z}^d} \frac{F(|\bk|)}{c(\bk)} \cos(\pi
  \bk\cdot \bthe),\qquad\hbox{since}\ f(\bx)\ \hbox{is
  real}.
\end{eqnarray}
Let the expanded coefficients $G(\bk):=F(|\bk|)/c(\bk), \bk \in {\Z}^d$, hence
\begin{equation}\label{fcospiz}
 f(\cos(\pi \bz))=\dsum_{\bk\in {\Z}^d}
 G(\bk)\cos(\pi \bk \cdot \bz).
\end{equation}
To make the proof more clear,  define $\hat{G}(\bk)$ as the
following

\begin{eqnarray}\label{hatG}
 \hat{G}(\bk) &:=& \frac{1}{n} \dsum_{\bz\in P}f(\cos(\pi \bz))
 \cos(\pi\bk \cdot \bz)\nonumber\\
 &=& \frac{1}{n} \dsum_{\bz\in P} \left[\dsum_{\bl\in {\Z}^d}
 {G(\bl)}
 \cos(\pi \bl \cdot \bz)\right]\cos(\pi\bk \cdot
 \bz)\nonumber\\
 &=& \frac{1}{2} \dsum_{\bl\in {\Z}^d}
 {G(\bl)} \frac{1}{n}
 \dsum_{\bz\in P} \big[\cos(\pi(\bl+\bk)\cdot \bz)
  + \cos(\pi(\bl-\bk)\cdot \bz)\big]\nonumber\\
  &=& \frac{1}{2} \dsum_{\bl\in {\Z}^d}
{G(\bl)}
  \Big[(-1)^{(\bl +\bk)\cdot \bh/(2n)}\delta_{(\bl+\bk)
  \cdot \bh\ \hbox{\small mod}\ 2n, 0}
  \nonumber\\
  &&\qquad \qquad \qquad
  +\; (-1)^{(\bl - \bk)\cdot \bh/(2n)}\delta_{(\bl-\bk)
  \cdot \bh\ \hbox{\small mod}\ 2n, 0}
   \Big],
\end{eqnarray}
where in the last equality, (\ref{shiftlatticeproperty}) have been
used.
 Let
\begin{eqnarray*}
 -\bl'&=&\bl +\bk,\quad\hbox{then}\
 \bl=-\bk-\bl';\\
 \bl''&=&\bl -\bk,\quad\hbox{then}\
 \bl=\bk+\bl''.
\end{eqnarray*}
Substitute these into the above expression for $\hat{G}(\bk)$.
Then

\begin{eqnarray}\label{Aliasing}
 \hat{G}(\bk) &=& \frac{1}{2} \left[
 \dsum_{\bl'\in L^{\bot}}
  {G(-\bk-\bl')}
 (-1)^{-\bl'\cdot \bh/(2n)} +
 \dsum_{\bl''\in L^{\bot}}
  {G(\bk+\bl'')}
 (-1)^{\bl''\cdot \bh/(2n)}\right]\nonumber\\
 &=& \dsum_{\bl\in L^{\bot}} {G(\bk + \bl)}
 (-1)^{\bl\cdot \bh/(2n)}\nonumber\\
 &=& {G(\bk)}
 + \dsum_{0\neq\bl\in L^{\bot}} {G(\bk + \bl)}
 (-1)^{\bl\cdot \bh/(2n)}.
\end{eqnarray}
In \eqref{fhatK}, $\hat{F}(\bk)$, the approximation of  $F(\bk)$, hence can be calculated by a sum of $\hat{G}(\bl)$ for some $|\bl|=\bk$.

How to choose ${\K}$ then becomes a very
important issue. ${\K}$ should have the following two properties.
\begin{enumerate}\label{prpt}
\item $F(\bx)$ should be small enough if
$\bk\in\Z^d\setminus{{\K}};$ \item $\hat{F}(\bk)$ should be a
good approximation to  ${F}(\bk)$ if
$\bk\in {\K}.$
\end{enumerate}

 According to the first property  of ${\K}$, one would like to
choose ${{\K}}$ in a way that it contains all important wavenumbers.
Hence, ${\K}$ should contain the wavenumbers near the origin and the
axes. Furthermore, the aliasing property listed in
\eqref{Aliasing} reminds us of the $\calV$ introduced in section 1
and motivates us to use corresponding nonnegative wavenumbers of $\calV_{2n}$
to construct the wavenumber ${\K}$,



Let
$${\calV}_{2n}^{+}=\{ \bk\in{\N}_0^d: \bk=|\bl|, \bl\in  {\calV}_{2n}\}.$$ 
The important wavenumber set $\K$ is then chosen as ${\calV}_{2n}^{+}$. The Chebyshev coefficient $F(\bk)$ for $\bk \in {\calV}_{2n}^{+}$ can be approximated by $$\hat{F}(\bk)=\sum_{|\bl|=\bk,\\\bk\in {\calV}_{2n}}\hat{G}(\bl).$$

The approximation of $f(\bx)$  then can be formulated by
\begin{eqnarray}\label{fhat1}
\hat{f}(\bx)
&=& \dsum_{\bk\in {\calV}_{2n}^+}\left(\dsum_{\substack{|\bk'|=\bk\\\bk'\in {\calV}_{2n}}}\hat{G}(\bk')\right)T_{\bk}(\bx),
\end{eqnarray}
The
two-dimensional ${\calV}_{2n}^+$ is drawn in Fig.~3.2, which shows
that ${\calV}_{2n}^+$ is of the desired shape containing almost
all the small wavenumbers.

\begin{figure}[ht]\label{wvnK^+}
\begin{center}
\includegraphics[width=13cm,height=13cm]{eps/wavenumber.eps}
\caption{Two-dimensional wavenumber space ${\calV}_{2n}^+$ for
$n=17,72,305,1292$ respectively.}
\end{center}
\end{figure}

Due to the unique structure of node set of lattice, we can use the fast cosine transformation to calculate $\hat{G}(\bk)$ for $\bk \in {\cal V}_{2n}$.
Suppose $\bk_i \in {\cal V}_{2n}$ and $\bk_i\cdot\bh = i \mod 2n$, and $f_j={f}(\bx_j),\bx_j \in P.$
According to (\ref{hatG}) we can compute $\hat{G}(\bk_i)$ as
\begin{equation*}
n\hat{G}(\bk_i)=\mathop \sum \limits_{j = 0}^{n-1}  f_j \cos \left((2j+1)\frac{\bk_i\cdot\bh}
{{2n}}\pi \right)=(-1)^{(\bk_i\cdot\bh - i)/(2n) } \mathop \sum \limits_{j = 0}^{n-1}  f_j \cos \left(i\frac{{2j + 1}}
{{N}}\pi \right).
\end{equation*}
We therefore can use discrete cosine transform (MATLAB DCT subroutine ) to compute the coefficients.


\section{Error Analysis}
The approximation error depends on how close $\hat F(\bk)$ is to
$F(\bk)$ for $\bk$ in the wavenumber space and how fast $F(\bk)$
decays with $\bk$ as $\bk$ gets away from the origin. To minimize
the approximation error, one needs to understand how it depends on
$\calP$ and $\calV$. The errors in the algorithm come from two
sources. One source is the wavenumbers not contained in $\calV$.
The Chebyshev coefficients of these wavenumbers are approximated by
zero. Thus, one would like to choose $\calV$ so that it contains
all important wavenumbers.  For the wavenumbers contained in
$\calV$, the estimates of the Chebyshev coefficients are
contaminated by the effects of aliasing.  Thus, we want to choose
$\K$ to contain all the small wavenumbers and no wavenumbers
aliased with each other. Aliasing is a phenomenon for both grids
and rank-1 lattices.  Two wavenumbers are aliased if their
difference is contained in the dual lattice, $L^\perp$.  Thus, one
may judge the quality of a lattice by its corresponding dual
lattice.  For well-chosen rank-1 lattices, $\K$ can contain almost
all wavenumbers $\bk=(k_1, \ldots, k_d)$ provided that
$\tilde{r}_\bgam(\bk)< C n^{1/2-\epsilon}$. 
%However, for a grid, $\K$
%cannot contain wavenumbers $\bk=(k_1, \ldots, k_d)$ with any
%$\abs{k_j} > n^{1/d}$, a constraint that becomes restrictive as
%$d$ increases. 
See, e.g., Niederreiter \cite{Niederreiter1992},
Sloan \& Joe \cite{SloJoe92}, and Li \& Hickernell \cite{LiHic03}
for a discussion on the quality measures for lattices.
 \smallskip


Define the weighted $L_2$ norm as
\begin{equation} \label{weiL2}
\|f\|_{L_2^{\omega}(\mathcal X)}:= \left(\dint_{\mathcal
X}[f(\bx)]^2 \omega (\bx)d\,\bx\right)^{1/2}.\end{equation}
The
asymptotic bound for \eqref{fhat1} in the weighted $L_2$ norm can
then be derived for large enough $n$'s as in the following.

\vskip5mm   \begin{Theorem}\label{C4err} Let $f \in
{\calF}_{\alpha,p,\bgam}$ and $\hat{f}$ be as in \eqref{fhat1}.
Then
\begin{eqnarray}
 \norm{f-\hat{f}}_{L_2^{\omega}(\mathcal X)} \leq 2\pi^{d/2}\widetilde{D}_{\alpha ,q,\bgam}(\calV_{2n})
  V_{\alpha,p,\bgam}(f),\quad \frac{1}{p}+\frac{1}{q}=1.\end{eqnarray}Thus, there exist rank-$1$
integration lattices such that
\begin{eqnarray}\label{eqn:C4err}
 \norm{f-\hat{f}}_{L_2^{\omega}(\mathcal X)}=\mathcal{O}\left( n^{-\frac{\alpha+ 1/q}{2-1/(\alpha q)}+\epsilon}\right).
\end{eqnarray}
The implied factor in the big $\mathcal{O}$-notation is dependent
on $\alpha$ and $d.$
\end{Theorem}
 \begin{proof} By the definition of
$\hat{f}(\bx)$ and the aliasing property (\ref{Aliasing}), it is
clear that the errors in the algorithm come from two sources. One
source is the wavenumbers not contained in ${\calV}_{2n}^+$. The
Chebyshev coefficients of these wavenumbers are approximated by
zeros. For the wavenumbers contained in ${\calV}_{2n}^+$, the
estimates of the Chebyshev coefficients are contaminated by the
effects of aliasing and the corresponding expanded coefficients whose wavenumbers not in ${\calV}_{2n}$. To proceed, there are still two facts to be  mentioned. One is  $F(\bk)=\sum_{|\bk'|=\bk,\atop\bk'\in {\Z}^d }{G(\bk')}$ with  $\#\{\bk'\in {\Z}^d : |\bk'|=\bk,\bk\in {\N}^d_{0}\}=c(\bk)$, the other is   the famous inequality: arithmetic mean is less or equal than the quadratic mean. Thus  $|F(\bk)|^2/c(\bk) \le \sum_{|\bk'|=\bk,\atop\bk'\in {\Z}^d }\abs{G(\bk')}^2$.  It hence follows that

\begin{eqnarray}
&&\norm{f-\hat{f}}_{L_2^{\omega}(\mathcal X)}\\
& =& \pi^{d/2}\left(\dsum_{\bk\in{\calV}_{2n}^+}\frac{1}{c(\bk)}{{\abs{F(\bk)-\sum_{|\bk'|=\bk,\atop\bk'\in {\calV}_{2n}}\hat{G}(\bk')}}^2}
+ \dsum_{\bk\notin{\calV}_{2n}^+,\atop \bk \in
{\N}^d_{0}}\frac{1}{c(\bk)}{\abs{F(\bk)}^2}\right)^{1/2}\nonumber\\
& =& \pi^{d/2}\left(\dsum_{\bk\in{\calV}_{2n}^+}\frac{1}{c(\bk)}{{\abs{\sum_{|\bk''|=\bk,\atop\bk''\in {\Z}^d}G(\bk'')-\sum_{|\bk'|=\bk,\atop\bk'\in {\calV}_{2n}}\hat{G}(\bk')}}^2}
+ \dsum_{\bk\notin{\calV}_{2n}^+,\atop \bk \in
{\N}^d_{0}}\frac{1}{c(\bk)}{\abs{F(\bk)}^2}\right)^{1/2}\nonumber\\
& \leq& \pi^{d/2}\left(\dsum_{\bk\in{\calV}_{2n}^+}\left(\dsum_{|\bk'|=\bk,\atop\bk'\in {\calV}_{2n}}\abs{G(\bk')-\hat{G}(\bk')}^2
+\sum_{|\bk'|=\bk,\atop\bk'\notin {\calV}_{2n}}\abs{{G}(\bk')}^2\right)+ \dsum_{\bk\notin{\calV}_{2n}^+,\atop \bk \in
{\N}^d_{0}}\dsum_{|\bk'|=\bk,\atop\bk'\in {\Z}^d}{\abs{G(\bk')}^2}\right)^{1/2}\nonumber\\
& \leq& \pi^{d/2}\left(\dsum_{\bk\in{\calV}_{2n}\atop\bk\in {\Z}^d}\abs{G(\bk)-\hat{G}(\bk)}^2
+\sum_{\bk\notin {\calV}_{2n}\atop\bk\in {\Z}^d}\abs{{G}(\bk)}^2\right)^{1/2}\nonumber\\
& \leq& \pi^{d/2}\left(\dsum_{\bk\in{\calV}_{2n}}\dsum_{0\neq\bl\in
L^\bot} \abs{G(\bk+\bl)} +
\dsum_{\bk\notin{\calV}_{2n}} \abs{G(\bk)}\right)\nonumber\\
&\leq& 2\pi^{d/2}
\left(\dsum_{{\Z}^d \ni \bk\notin{\calV}_{2n}} |G(\bk)[\tilde{r}_\bgam(\bk)]^{\alpha}|^p\right)^{\frac{1}{p}}
\norm{[\tilde{r}_\bgam(\bk)]^{-\alpha}1_{{\bk\notin
{\calV}_{2n}}}}_q\nonumber\\
&\le&  2\pi^{d/2}\left(\dsum_{\bk\in{\Z}^d} {\frac{F(|\bk|)}{c(\bk)}[\tilde{r}_\bgam(\bk)]^{\alpha}}^p\right)^{\frac{1}{p}}
\norm{[\tilde{r}_\bgam(\bk)]^{-\alpha}1_{{\bk\notin
{\calV}_{2n}}}}_q\nonumber\\
&=&  2\pi^{d/2}\left(\dsum_{\bk\in{\N}^d_0} \dsum_{|\bl|=\bk}\frac{\abs{F(|\bl|)}^p[\tilde{r}_\bgam(\bl)]^{\alpha p}}{[c(\bl)]^p}\right)^{\frac{1}{p}}
\norm{[\tilde{r}_\bgam(\bk)]^{-\alpha}1_{{\bk\notin
{\calV}_{2n}}}}_q\nonumber\\
&\leq&  2\pi^{d/2}\left(\dsum_{\bk\in{\N}^d_{0}} \abs{{F(\bk)}[\tilde{r}_\bgam(\bk)]^{\alpha}}^p\right)^{\frac{1}{p}}
\norm{[\tilde{r}_\bgam(\bk)]^{-\alpha}1_{{\bk\notin
{\calV}_{2n}}}}_q\nonumber\\
& \leq &
2\pi^{d/2}
  V_{\alpha,p,\bgam}(f)\widetilde{D}_{\alpha ,q,\bgam}(\calV_{2n}).
\end{eqnarray}
With the equality (\ref{eqnBCB}), the proof is
completed.\end{proof}



\vskip8mm  \section{Tractability and Strong Tractability}

From the proof of Theorem \ref{C4err}, the implied factor in the
big $\mathcal{O}$-notation of \eqref{eqn:C4err} is  increasing
exponentially with increasing $d.$ Hence when dimension goes to
infinity, the constants would go to infinity too. How to eliminate
the effect of dimension or constrain the increase of the factor to
polynomial rate shall be studied in this section.

Suppose that there exists an algorithm $s(f)$ to approximate $f$
for the function $f$ in the function space $\mathcal F.$  The
worst case error of the algorithm $s(f)$ using $n$ sample points
is
$$\err\hspace{0.01cm}(n,\mathcal F)=
\sup\limits_{{\norm{f}}_{\mathcal F} \leq 1 }\norm{s(f)-f}.$$ The
initial error is
\begin{equation}\label{initialerror}\err\hspace{0.01cm}(0,\mathcal F)=
\sup\limits_{{\norm{f}}_{\mathcal F} \leq 1
}\norm{f}.\end{equation} For $\varepsilon\in(0,1)$ and $d\ge 1,$
let $n^{wor}(\varepsilon,\mathcal F)$ be the minimal number of
function samples needed to solve the problem, i.e.,
$$n^{wor}(\varepsilon,\mathcal F):= \min\{n: \err\hspace{0.01cm}(n,\mathcal F)\le \varepsilon\err\hspace{0.01cm}(0,\mathcal F) \}.$$
The algorithm for weighted Banach space is  tractable  iff there
are non-negative numbers $C, p$ and $q$ such that
$$n^{wor}(\varepsilon,\mathcal F)\le C
\varepsilon^{-p}d^{q},\quad \forall \varepsilon\in(0,1), \forall
d\ge 1.$$ The  algorithm is  strongly tractable iff $q=0.$ In
other words,  the algorithm for the weighted Banach space is
tractable iff there are non-negative numbers $C, a$ and $b$ such
that
$$\err\hspace{0.01cm}(n,\mathcal F)\le C\err\hspace{0.01cm}(0,\mathcal F)
n^{-a}d^{b},\quad  \forall d\ge 1.$$ The  algorithm is strongly
tractable iff $b=0.$

The tractability and strong tractability of the algorithm proposed
by \eqref{fhat1} and \eqref{Aliasing} will be discussed here. The
function space is now constrained to $\mathcal F_{\alpha,p,\bgam}$
and $\norm{\cdot}$ is defined in \eqref{weiL2}.

Obviously, the norm of  $f\equiv 1,$ $V_{\alpha,p,{\bgam}}(f)=1,$
since the Chebyshev coefficient of the constant function $f$ is
one at the origin and  zero otherwise. It can be obtained that
$\norm{f}_{L_2^{\omega}(\mathcal X)}=\pi^{d/2}.$ The initial error
defined in \eqref{initialerror} is greater than $\pi^{d/2}.$

According to the Theorem \ref{C4err},  we  can obtain the following
theorem. \vskip5mm
\begin{Theorem}\label{trc4errrst}
Let $f \in {\calF}_{\alpha,p,{\bgam}},$
 $1/q\leq v<\alpha$ and $n$ is large enough. Then a
generating vector of a rank-$1$ integration lattice can be
obtained  by component-by-component construction such that
 if $\dsum_{j=1}^{+\infty}
\gamma_j^{\alpha/v}<+\infty,$  then
\begin{eqnarray}
\sup_{\|f\|_{\mathcal F_{\alpha,p,{\bgam}}}\leq 1}
\norm{f-\hat{f}}_{L_2^{\omega}(\mathcal
X)}[\err\hspace{0.01cm}(0,\mathcal F_{\alpha,p,{\bgam}})]^{-1}
=\mathcal{O}\left(n^{-\frac{v-1/q}{ 2-1/(vq)}}\right);
\end{eqnarray}if
$\sum\limits_{i=1}^{\infty}\gamma_j^{\alpha/v}=\infty,$
but
$a:=\lim\sup_{d\rightarrow\infty}\sum\limits_{i=1}^{d}\gamma_j^{\alpha/v}/\log(d+1)<\infty,$
then
\begin{eqnarray}\sup_{\|f\|_{\mathcal F_{\alpha,p,{\bgam}}}\leq 1}
\norm{f-\hat{f}}_{L_2^{\omega}(\mathcal
X)}[\err\hspace{0.01cm}(0,\mathcal F_{\alpha,p,{\bgam}})]^{-1}
=\mathcal{O}\left(d^{b}n^{-\frac{v-1/q}{ 2-1/(vq)}}\right) .
\end{eqnarray}
The implied factor in the big $\mathcal{O}$-notation is
independent of $d$. The exponent $b$ is any positive number
greater than $2av\bzeta\big(\alpha / v \big)$ and
$\frac{v-1/q}{2-1/(qv)}$ can be arbitrarily close to
$({\alpha-\frac{1}{q}})/({2-\frac{1}{\alpha q}}).$
\end{Theorem}


\vskip8mm  \section{Numerical Experiments}

The $8$-dimensional borehole function described by
\cite{MorMitYlv93}
\begin{equation}\label{borehole}
\displaystyle\frac{2\pi
T_u(H_u-H_l)}{\log(\frac{r}{r_w})\left(1+\frac{2L
T_u}{\log(\frac{r}{r_w})r_w^2K_w}+\frac{T_u}{T_l}\right)}
\end{equation}
is used  as the test function. The borehole function is a model
for the flow rate of water from an upper to a lower aquifer
through an impermeable rock with a borehole through that layer
connecting them. Here $r,r_w$ are radii of the borehole and the
surrounding basin respectively;
 $T_l,T_u$ are  transmissivities of the upper and lower aquifers;
 $H_l,H_u$ are potentiometric heads;
 $L$ represents the length of the borehole;
 $K_w$ denotes the conductivity.

 The borehole function of equation (\ref{borehole}) is
investigated over the following ranges:
 \begin{eqnarray*}r_w &
&\in [0.05, 0.15]\, m;\\ r && \in [100, 50000]\, m;\\ T_u && \in
[63070, 115600] m^3/yr;\\ T_l &&\in [63.1, 116] \, m^3/yr;\\ H_u
&&\in
[990, 1110]\, m;\\ H_l &&\in [700, 820]\, m ;\\L &&\in [1120, 1680] \,m;\\
K_w &&\in [9855, 12045] \, m/yr.
\end{eqnarray*}

\begin{figure}[ht]
\begin{center}
\includegraphics[width=12cm]{eps/borehole1.eps}
\caption{Root relative squared  error for $8$-dimensional Borehole function.}
\end{center}
\end{figure}

 The integer vector  $(1 ,17797,\cdots,17797^{d-1} )^T \mod 2n $, an extensible rank-1 lattice design introduced in  \cite{HicHong2000},
is used  as $\bh$ in the numerical experiments. The corresponding important wavenumber sets are calculated by exhaustive search  and the weight in each dimension is chosen to be $1$. 

For an algorithm ${\cal A}$, the root relative squared errors (RRSE) is defined by
$$\frac{\sqrt{\sum_{i=1}^M ({\cal A}(f)(\bx_i)-f(\bx_i)})}{\sqrt{\sum_{i=1}^M (f(\bx_i)-\frac{1}{M}\sum_{j=1}^M f(\bx_j)) }}.$$ 
In our experiments, $\{\bx_i\}$ are chosen to be $250$ random  points. And the algorithm ${\cal A}$ are the proposed method in the article, and for comparison, the spectral approximation used Chebyshev-Gauss-Lobatto (CGL) grids  with  the same number of projections in each dimension.
Fig.~6.1 shows the simulation results of RRSE. The
circle represents the error of the rank-1 lattice approximation
in \eqref{fhat1} and the star is of spectral grids method. From Fig.~6.1, our method outperforms the spectral CGL approximation. The convergence rate of our method is about
$\mathcal{O}(n^{-1}).$ Higher convergence rate can be achieved  if
better generators and more regular test functions are used.
 \clearpage





%%%%%%%%%%%%%%%%%%%%%%%%%%%% Chapters End %%%%%%%%%%%%%
%%%%%%%%%%%%%%%%%%%%%%%%%%%% Chapters End %%%%%%%%%%%%%
%%%%%%%%%%%%%%%%%%%%%%%%%%%% Chapters End %%%%%%%%%%%%%


 \section*{Acknowledgements} The authors wish to thank Professors
Jie Shen and Tao Tang for their valuable discussions and thank the
two anonymous referees for their constructive suggestions which
lead to a greatly improved paper.


\begin{thebibliography}{10}
\bibitem{BernardiM1997} {\sc C. Bernardi and Y. Maday}, {\em
Spectral Methods}, in Handbook of Numerical Analysis, VOL.~V,
Techniques of Scientific Computing (Part~2), P.~G. Ciarlet and
J.~L. Lions, eds., North-Holland, Amsterdam, 1997, pp.~209--485.

\bibitem{BernardiM1989} {\sc C. Bernardi and Y. Maday}, {\em
Properties of some weighted Sobolev spaces and application to
spectral approximations}, SIAM J. Numer. Anal., 26 (1989),
pp.~769--829.

\bibitem{Boyd2000} {\sc J.~P. Boyd}, {\em Chebyshev and Fourier
Spectral Methods}, Dover, New York, 2000.

\bibitem{CanutoHQZ1988} {\sc C. Canuto, M.~Y. Hussaini, A.
Quarteroni, and T.~A. Zang}, {\em Spectral Methods in Fliud
Dynamics}, Spring-Verlag, New York, 1988.

\bibitem{CanutoQ1981} {\sc C. Canuto and A. Quarteroni}, {\em
Spectral and pseudo-spectral methods for parabolic problems with
nonperiodic boundary conditions}, Calcolo, 18 (1981),
pp.~197--218.

\bibitem{Don1994} {\sc W.~S. Don and D. Gottlieb}, {\em The
Chebyshev-Legendre method: Implementing Legendre methods on
Chebyshev points}, SIAM J. Numer. Anal., 31 (1994),
pp.~1519--1534.

\bibitem{Fornberg1995} {\sc B. Fornberg}, {\em A Practical Guide
to Pseudospectral Methods}, Cambridge Monographs on Applied and
Computational Mathematics, Cambridge University Press, Cambridge,
1995.

\bibitem{Funaro1986} {\sc D. Funaro}, {\em A multidomain spectral
approximation of elliptic equations}, Numer. Meth. PDEs, 2 (1986),
pp.~187--205.

\bibitem{Gottlieb1984} {\sc D. Gottlieb, M.~Y. Hussaini and S~.A.
Orszag}, {\em Introduction: Theory and Application of Spetral
Methods}, in Spectral Methods for Partial Differential Equations,
R~.G. Voigt, D. Gottlieb and M.~Y. Hussaini, eds., SIAM,
Philadelphia, 1984.

\bibitem{GottliebO1977} {\sc D. Gottlieb and S.~A. Orszag}, {\em
Numerical analysis of spectral methods: theory and applications},
CBMS-NSF Regional Conference Series in Applied Mathematics, SIAM,
Philadelphia, 1977.

\bibitem{Hai-Z1979} {\sc D.~B. Haidvogel and T. Zang},
{\em The accurate solution of Poisson equation by expansion in
Chebyshev polynomials}, J. Comp. Phy, 30 (1979), 167--180.


\bibitem{Hickernell1998}
{\sc F.~J. Hickernell}, {\em Lattice rules: How well do they
measure up?}, in Random and Quasi-Random Point Sets (P. Hellekalek
and G. Larcher, eds.), Lecture Notes in Statistics, vol. 138,
Springer-Verlag, New York, 1998, pp.~109--166.

\bibitem{HickernellNie2003} {\sc F.~J. Hickernell and H.
Niederreiter}, {\em The existence of good extensible rank-1
lattices}, J. Complexity, 19 (2003), pp.~286--300.

\bibitem{HuaWang1981} {\sc L.~K. Hua and Y. Wang}, {\em
Applications of Number Theory to Numerical Analysis},
Springer-Verlag and Science Press, Berlin and Beijing, 1981.

\bibitem{KuoSloWoz05a} {\sc F.~Y. Kuo and I.~H. Sloan and H.
Wo\'zniakowski}, {\em Lattice rules for multivariate approximation
in the worst case setting},  in {M}onte {C}arlo and Quasi-{M}onte
{C}arlo Methods 2004 (H. Niederreiter and D. Talay, eds.),
Springer-Verlag, Berlin, 2005.

\bibitem{LiHic03}
{\sc D. Li and F.~J. Hickernell}, {\em Trigonometric spectral
methods on lattices}, in Recent Advances in Scientific Computing
and Partial Differential Equations (S. Y. Cheng, C.-W. Shu, and T.
Tang, eds.), AMS Series in Contemporary Mathematics, vol. 330,
American Mathematical Society, Providence, Rhode Island, 2003,
pp.~121--132.

\bibitem{MasonH2003} {\sc J.~C. Mason and D. Hanscomb}, {\em
Chebyshev Polynomials}, CRC Press LLC, Florida, 2003.

\bibitem{Mercier1989} {\sc B. Mercier}, {\em An Introduction to
the Numerical Analysis of Spectral Methods}, Springer-Verlag, New
York, 1989.

\bibitem{MorMitYlv93} {\sc  M.~D. Morris,  T. ~J. Mitchell,
and D. Ylvisaker }, {\em Bayesian design and analysis of computer
experiments: Use of derivatives in surface prediction},
Technometrics, 35 (1993), pp.~243--255.

\bibitem{Niederreiter1992} {\sc H. Niederreiter}, {\em Random
Number Generation and Quasi-Monte Carlo Methods}, CBMS-NSF
Regional Conference Series in Applied Mathematics, SIAM,
Philadelphia, 1992.

\bibitem{ShenT07} {\sc J. Shen and T. Tang},
Spectral and High-Order Methods with Applications, Science Press,
2007.

\bibitem{SloJoe92}
{\sc I.~H. Sloan and S. Joe}, {\em Lattice Methods For Multiple
Integration}, Academic Press, San Diego, 1992, pp.~146--147.

\bibitem{Trefethen2000} {\sc L.~N. Trefethen}, {\em Spectral
Methods in Matlab}, SIAM Philadelphia, 2000.

\bibitem{Traub88} {\sc J.~F. Traub, G.~W. Wasilkowski and H. Wozniakowski}, {\em Information-based Complexity}, Academic Press, 1988.

\bibitem{ZenEtal05a} {\sc X. Zeng, K.~T. Leung and F.~J. Hickernell},
  {\em Error analysis of splines for periodic problems using
  lattice designs}, in {M}onte {C}arlo and
Quasi-{M}onte {C}arlo Methods 2004 (H. Niederreiter and D. Talay,
eds.), Springer-Verlag, Berlin, 2005.

\bibitem{Zengthesis} {\sc X. Zeng},
  {\em Integration and Approximation Using Lattice Designs}, 2008.

 \bibitem{HicHong2000}
{\sc F. ~J. Hickernell, H. ~S. Hong, P. L'Ecuyer  and  C. Lemieux},
{\em Extensible lattice sequences for quasi-{M}onte {C}arlo quadrature},
SIAM J. Sci. Comput., 22(2000), pp.~1117-1138
\end{thebibliography}

\end{document}
